\documentclass[a4paper]{article}
\usepackage[backend=biber, style=alphabetic, sorting=ynt]{biblatex}
\usepackage[utf8]{inputenc}
\usepackage{graphicx}
\usepackage{amsmath}
\usepackage{listings}
\usepackage[bottom=2.0cm,top=2.0cm,left=2.0cm,right=2.0cm]{geometry}
\usepackage[portuges]{babel}
\usepackage{indentfirst}
\usepackage{hyperref}  %%%%
\hypersetup{colorlinks,citecolor=black,filecolor=black,linkcolor=black,urlcolor=black} 
\addbibresource{referencias.bib}

\begin{document}
\title{Relatório Experimental}

\begin{titlepage}
	\begin{center}
		\begin{figure}[htb!]
			\begin{flushleft}
				\includegraphics[width=3.9cm]{imagens/logo_unb.png}
			\end{flushleft}
		\end{figure}
        \vspace{-2.5cm}
        \hspace{2.1cm}\Large{\textbf{Universidade de Brasília}}\\
        \hspace{2.1cm}\Large{Faculdade UnB Gama – FGA}\\
        \hspace{2.1cm}\Large{Engenharia de Software}\\
        
        \vspace{200pt}
        
        \LARGE{\textbf{Chat em linguagem C usando o select()}}\\ 
        %\Large{Chat com uso do select()}\\ 
        
        \vspace{160pt}
            
        \vspace{30pt} 
        \hfill Giulia Lobo Barros \hspace{20pt} \\
        \hfill Igor Queiroz Lima  \hspace{20pt} \\
        \hfill Nathalia Lorena Cardoso Dias \hspace{20pt}\\
        \hfill Thiago Aparecido Lopes Santos \hspace{20pt}\\

        \vspace{25pt}
        \hfill {Professor:}\\
        \hfill Fernando William Cruz\\
        
        
        \vspace{\fill}
        \LARGE \bf{\today}
          
	\end{center}
\end{titlepage}

\newpage

%\Large\tableofcontents
%\thispagestyle{empty}
             %%%%%
%\newpage
\pagenumbering{arabic}
\large

\section{Introdução}

Este projeto faz parte da disciplina Fundamentos de Redes de Computadores, da Universidade de Brası́lia, campus Gama (UnB FGA) orientada pelo professor Fernando William Cruz. O projeto tem como objetivo, permitir a compreensão da arquitetura de aplicações de rede (segundo arquitetura TCP/IP) que envolvam gerência de diálogo. Para isso, os alunos devem construir uma aplicação que disponibilize salas de bate-papo virtuais, nas quais os clientes podem ingressar e interagir.

\section{Objetivos}

O objetivo do projeto é criar salas de bate-papo virtuais que atendam
às seguintes especificações:

 \begin{enumerate}
   \item A criação de salas virtuais de bate-papo com nome da sala e limite de participantes.
   \item Listar participantes de uma determinada sala.
    \item Permitir ingresso de clientes, com um identificador, em uma sala existente, de acordo com o limite admitido para a sala.
    \item Saída de clientes de uma sala em que estava participando.
    \item Diálogo entre os clientes das salas.
    \item Sugere-se que o servidor contenha apenas funções de gerenciamento de diálogo (fazendo o repasse das
    interações, de acordo com as salas) e funções administrativas mínimas (inclusão/exclusão de salas, ingresso/saída de usuários de uma sala, etc.) para simplificar o projeto.
    \item O diálogo entre cliente e servidor deve ser feito usando a System call select() para organização dos diálogos. Os clientes, por sua vez, devem conhecer o endereço do servidor a fim de se registrarem para participação em diálogos.
\item A aplicação deve ser construída em linguagem C.

\end{enumerate}

\section{Metodologia}

Este projeto foi desenvolvido na linguagem C e faz o uso das bibliotecas socket e select. Foi criado um arquivo chamado "socket.c", que gerencia toda a conexão entre servidor e clientes. Também foi criado um arquivo "room.c" que é responsável pela criação e gerência das salas. Além disso, foi criado um arquivo "command.c" onde são especificados e tratados os comandos que o usuário pode digitar. Por fim, o arquivo "main.c" que invoca as funções principais para o funcionamento do chat.

\subsection{Organização da equipe}

Nos reunimos inicialmente para entender o problema e
buscar a melhor forma de trabalharmos juntos na solução. Nossas reuniões ocorreram via Discord, desenvolvemos o código na ferramenta VSCode e usamos a extensão Live Share da própria ferramenta para podermos contribuir simultaneamente no código. Em algumas reuniões trabalhamos em pares, pois nem todos os membros estavam disponíveis nos mesmos dias e horários. Por fim, na elaboração do relatório e da apresentação todos os membros estavam disponíveis.

\section{Descrição da solução}
\subsection{Socket}

Este código implementa um servidor TCP básico usando sockets e o select em linguagem C. Ele cria um servidor que aguarda conexões de clientes, recebe mensagens dos clientes, executa comandos e envia respostas de volta aos clientes. Ele contém várias funções responsáveis por diferentes etapas do processo de comunicação entre o servidor e os clientes. No início do código, são declaradas algumas variáveis globais que serão utilizadas ao longo do programa, como o socket do servidor, conjuntos de sockets e o valor máximo de descritores de sockets.

\subsection{Room}

Este código define duas estruturas de dados: "person t" e "room t", que representam uma pessoa e uma sala, respectivamente. Suas funções permitem que os clientes se conectem a salas de chat, interajam com outras pessoas presentes na mesma sala e enviem mensagens para a sala. O código mantém o controle das salas disponíveis, das pessoas conectadas e de suas respectivas salas.

\subsection{Command}

O código fornecido implementa o sistema de comandos para o chat. Ele define várias funções que são mapeadas para comandos específicos, bem como uma estrutura de dados chamada "command t" que associa os comandos às suas respectivas funções. Além disso, existe a função "execute command", que recebe um comando e o cliente associado e executa a função correspondente ao comando.

\subsection{Main}

Este código é iniciado abrindo um socket, configurando-o para reutilizar o endereço e associando-o a um IP e porta específicos. Em seguida, o servidor entra em um loop principal, onde aguarda conexões de clientes. Durante esse tempo, as salas de chat são inicializadas e o servidor está pronto para atender várias solicitações de clientes simultaneamente. O programa continuará em execução indefinidamente, até que seja encerrado manualmente. A "Main" estabelece a base do servidor de chat, cuidando das configurações de rede e da espera por conexões de clientes.

\section{Conclusão}

Tivemos algumas dificuldades iniciais de entendimento com função select, buscamos algumas outras fontes de conhecimento como vídeos para compreender melhor o uso do select além da referência sugerida pelo professor. No entanto, alguns colegas tiveram uma compreensão melhor sobre o select, e os pareamentos contribuiram para o compartilhamento e nivelamento de conhecimento.

\subsection{Nathalia}

Este projeto foi mais enriquecedor em termos de aquisição de conhecimento do que os outros projetos. Permitiu uma compreensão prática sobre o select, permitiu a integração dos conceitos de redes de computadores com outros conhecimentos adquiridos no curso de engenharia de software. Além disso, as discussões durante as reuniões entre os membros do grupo e a prática de programação em dupla contribuiram para o compartilhamento de conhecimento.
NOTA: 10

\subsection{Thiago}

O projeto foi muito interessante, pois foi possível integrar os conhecimento adquirido durante as aulas, aplicados ao contexto de engenharia de software, e se utilizando da linguagem C. Nos ajudou a compreender, mesmo de maneira básica, recursos que utilizamos no dia a dia sem saber exatamente tudo o que acontece por baixo dos panos. 
NOTA: 10

\section{Referências}

The Linux man-pages project. select(2) - linux manual page. Acessado em 2023.
Disponível em: https://man7.org/linux/man-pages/man2/select.2.html. 

\end{document}

