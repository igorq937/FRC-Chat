\documentclass[a4paper]{article}
\usepackage[backend=biber, style=alphabetic, sorting=ynt]{biblatex}
\usepackage[utf8]{inputenc}
\usepackage{graphicx}
\usepackage{amsmath}
\usepackage[bottom=2.0cm,top=2.0cm,left=2.0cm,right=2.0cm]{geometry}
\usepackage[portuges]{babel}
\usepackage{indentfirst}
\usepackage{hyperref}  %%%%
\hypersetup{colorlinks,citecolor=black,filecolor=black,linkcolor=black,urlcolor=black} 
\addbibresource{referencias.bib}

\begin{document}
\title{Relatório Experimental}

\begin{titlepage}
	\begin{center}
		\begin{figure}[htb!]
			\begin{flushleft}
				\includegraphics[width=3.9cm]{imagens/logo_unb.png}
			\end{flushleft}
		\end{figure}
        \vspace{-2.5cm}
        \hspace{2.1cm}\Large{\textbf{Universidade de Brasília}}\\
        \hspace{2.1cm}\Large{Faculdade UnB Gama – FGA}\\
        \hspace{2.1cm}\Large{Engenharia de Software}\\
        
        \vspace{200pt}
        
        \LARGE{\textbf{Chat em linguagem C usando o select()}}\\ 
        %\Large{Chat com uso do select()}\\ 
        
        \vspace{160pt}
            
        \vspace{30pt} 
        \hfill Giulia Lobo Barros \hspace{20pt} \\
        \hfill Igor Queiroz Lima  \hspace{20pt} \\
        \hfill Nathalia Lorena Cardoso Dias \hspace{20pt}\\
        \hfill Thiago Aparecido Lopes Santos \hspace{20pt}\\

        \vspace{25pt}
        \hfill {Professor:}\\
        \hfill Fernando William Cruz\\
        
        
        \vspace{\fill}
        \LARGE \bf{\today}
          
	\end{center}
\end{titlepage}

\newpage

%\Large\tableofcontents
%\thispagestyle{empty}
             %%%%%
%\newpage
\pagenumbering{arabic}
\large

\section{Introdução}

Este projeto faz parte da disciplina Fundamentos de Redes de Computadores, da Universidade de Brası́lia, campus Gama (UnB FGA) orientada pelo professor Fernando William Cruz. O projeto tem como objetivo, permitir a compreensão da arquitetura de aplicações de rede (segundo arquitetura TCP/IP) que envolvam gerência de diálogo. Para isso, os alunos devem construir uma aplicação que disponibilize salas de bate-papo virtuais, nas quais os clientes podem ingressar e interagir.

\section{Objetivos}

O objetivo do projeto é criar salas de bate-papo virtuais que atendam
às seguintes especificações:

 \begin{enumerate}
   \item A criação de salas virtuais de bate-papo com nome da sala e limite de participantes.
   \item Listar participantes de uma determinada sala.
    \item Permitir ingresso de clientes, com um identificador, em uma sala existente, de acordo com o limite admitido para a sala.
    \item Saída de clientes de uma sala em que estava participando.
    \item Diálogo entre os clientes das salas.
    \item Sugere-se que o servidor contenha apenas funções de gerenciamento de diálogo (fazendo o repasse das
    interações, de acordo com as salas) e funções administrativas mínimas (inclusão/exclusão de salas, ingresso/saída de usuários de uma sala, etc.) para simplificar o projeto.
    \item O diálogo entre cliente e servidor deve ser feito usando a System call select() para organização dos diálogos. Os clientes, por sua vez, devem conhecer o endereço do servidor a fim de se registrarem para participação em diálogos.
\item A aplicação deve ser construída em linguagem C.

\end{enumerate}

\section{Metodologia}

Este projeto foi desenvolvido na linguagem C e faz o uso das bibliotecas socket e select. Foi criado um arquivo chamado "socket.c", que gerencia toda a conexão entre servidor e clientes. Também foi criado um arquivo "room.c" que é responsável pela criação e gerência das salas. Além disso, foi criado um arquivo "command.c" onde são especificados e tratados os comandos que o usuário pode digitar. Por fim, o arquivo "main.c" que invoca as funções principais para o funcionamento do chat.

\subsection{Organização da equipe}

\section{Descrição da solução}

\section{Conclusão}

\section{Referências}

\end{document}

